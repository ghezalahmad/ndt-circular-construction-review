\documentclass[11pt,a4paper]{article}

\usepackage[utf8]{inputenc}
\usepackage{amsmath,amssymb}
\usepackage{graphicx}
\usepackage{booktabs}
\usepackage{multirow}
\usepackage{array}
\usepackage{tabularx}
\usepackage{longtable}
\usepackage{hyperref}
\usepackage{cleveref}
\usepackage{siunitx}
\usepackage{xcolor}
\usepackage[margin=2.5cm]{geometry}
\usepackage{authblk}
\usepackage{natbib}
\usepackage{url}
\usepackage{adjustbox}
\usepackage{fvextra}
\usepackage{float}




%% Begin - PRISMA Package_-------------------------------
\usepackage{tikz}
\usetikzlibrary{arrows.meta,positioning,fit,shapes.misc,calc}
\usepackage{adjustbox} % for max width scaling
\usepackage{dblfloatfix} % or \usepackage{stfloats}
\usepackage[table]{xcolor} % for gray!5

\tikzset{
  prismaBox/.style={
    rectangle, rounded corners=2pt, draw,
    align=center, inner sep=6pt, minimum width=4.1cm, minimum height=1.1cm
  },
  prismaWide/.style={prismaBox, minimum width=9.0cm},
  prismaArrow/.style={-{Latex}, thick}
}

% One-line macro to set all counts
% Usage: \PrismaSet{DB}{OTH}{DEDUP}{SCREEN}{EXCLSCREEN}{FULL}{EXCLFULL}{QUAL}{QUANT}
\newcommand{\PrismaSet}[9]{%
  \def\nDB{\textbf{#1}}%
  \def\nOTH{\textbf{#2}}%
  \def\nDEDUP{\textbf{#3}}%
  \def\nSCREEN{\textbf{#4}}%
  \def\nEXCLSCREEN{\textbf{#5}}%
  \def\nFULL{\textbf{#6}}%
  \def\nEXCLFULL{\textbf{#7}}%
  \def\nQUAL{\textbf{#8}}%
  \def\nQUANT{\textbf{#9}}%
}
%% End - PRISMA Package_-------------------------------














\title{\textbf{From Detection to Design Value: A Systematic Review of Non-Destructive Testing Capabilities for Circular Construction}}

\author[1]{Ghezal Ahmad Jan Zia}
\author[1,2]{Benjamín Moreno Torres}
\author[1,3]{Christoph Völker\thanks{Corresponding author: christoph.voelker@iteratec.com}}
\author[1,2]{Sabine Kruschwitz}

\affil[1]{Bundesanstalt für Materialforschung und -prüfung (BAM), Berlin, Germany}
\affil[2]{Technische Universität Berlin, Berlin, Germany}
\affil[3]{iteratec GmbH, Hamburg, Germany}

\date{January 2026}

\begin{document}

\maketitle

\begin{abstract}
Circular construction requires reliable assessment of existing structural elements for reuse. While non-destructive testing (NDT) methods can detect deterioration and estimate material properties, their capability to deliver design-code-compliant characteristic values remains poorly characterized. This systematic review evaluates whether NDT can bridge the gap between unknown in-situ conditions and the quantified material properties required for structural design. Following PRISMA guidelines, we analyzed 227 studies published between 2014--2024 across four structural material classes (reinforced concrete, steel, timber, masonry) and five assessment tasks (deterioration detection, strength estimation, defect characterization, moisture assessment, geometry verification). We constructed a Capability Matrix classifying 47 material-task-method combinations by validation status. Our analysis reveals that only 21\% achieve field-validated capability with documented uncertainty budgets sufficient for design value derivation. The critical constraint is not detection capability but the absence of validated uncertainty propagation frameworks---the methodology to transform measurements into EN~1990-compliant characteristic values. We propose that enabling circular construction requires a paradigm shift from detection-focused NDT toward characterization with quantified reliability, supported by standardized metadata ontologies that link measurement provenance to design decisions. This reframing identifies the binding constraint as metrological rather than technological, redirecting research priorities toward validation of existing methods rather than development of new sensors.
\end{abstract}

\noindent\textbf{Keywords:} non-destructive testing; circular construction; structural assessment; uncertainty quantification; material reuse; capability assessment

%=============================================================================
\section{Introduction}
\label{sec:intro}
%=============================================================================

\subsection{The Information Gap in Circular Construction}
\label{sec:intro:gap}

The construction sector accounts for approximately 40\% of global energy consumption and 36\% of CO$_2$ emissions, while generating over 35\% of solid waste in industrialized nations \citep{unep2020}. Construction and demolition waste represents 25--30\% of total waste in Europe, yet the average building lifespan rarely exceeds 39 years---often only 25--30 years in European contexts \citep{reincarnate2024}. This disparity between resource consumption and service life extraction creates a compelling case for circular construction: extending service life, enabling component reuse, and recovering materials at end of life.

The European Union's Circular Economy Action Plan identifies construction as a priority sector, mandating 70\% recovery of construction and demolition waste and proposing enhanced circularity requirements through the Construction Products Regulation revision \citep{eu2020ceap}. However, while policy frameworks advance rapidly, a fundamental technical barrier remains: \emph{we do not know what we have}.

New construction relies on controlled material specifications, quality assurance during production, and design calculations based on guaranteed characteristic values. Reuse of existing structural elements inverts this information flow: engineers must derive material properties post-hoc from structures with unknown production quality, variable degradation, and uncertain service history. This information asymmetry---knowing what is required but not what is available---constitutes the primary technical barrier to scaled component reuse.

\subsection{From Detection to Characterization: A Necessary Paradigm Shift}
\label{sec:intro:paradigm}

Non-destructive testing has been applied to structural assessment for decades, but its traditional role differs fundamentally from circular construction requirements. Conventional NDT practice focuses on \emph{detection}: identifying corrosion, locating defects, mapping deterioration. The output is typically qualitative (present/absent, good/fair/poor) or comparative (worse than reference, better than threshold). This suffices for maintenance planning, where the decision is \emph{when} to intervene, not \emph{whether} an element can carry specified loads in a new application.

Circular construction demands a different output: \emph{characteristic values}. Structural design per EN~1990 requires material properties defined as statistical fractiles with specified confidence---typically the 5\% fractile with 75\% confidence \citep{en1990}. The design resistance $R_d$ derives from:
\begin{equation}
R_d = \frac{R_k}{\gamma_M} = \frac{f_k \cdot A}{\gamma_M}
\label{eq:design}
\end{equation}
where $f_k$ is the characteristic material strength, $A$ the cross-sectional area, and $\gamma_M$ the partial factor accounting for material and model uncertainties. For reuse assessment, both $f_k$ and $A$ must be determined in-situ with quantified uncertainty that propagates through to $R_d$.

This requirement exposes a fundamental gap: most NDT methods were developed for detection, not characterization. They can identify \emph{that} corrosion exists but not \emph{how much} section loss has occurred with ±X\% uncertainty. They can indicate \emph{that} strength is approximately Y~MPa but not derive characteristic values compliant with EN~13791 \citep{en13791}. The paradigm shift required is from:
\begin{quote}
\emph{``Can we detect the problem?''} $\rightarrow$ \emph{``Can we quantify the property with known reliability?''}
\end{quote}

This reframing transforms the research question from technological (better sensors, higher resolution) to metrological (validated uncertainty, traceable calibration, documented detection limits).

\subsection{Hypothesis and Research Questions}
\label{sec:intro:rq}

We propose the following central hypothesis:

\begin{quote}
\textbf{Hypothesis:} Non-destructive testing can bridge the gap between unknown in-situ material conditions and the characteristic values required for structural design, but only if the paradigm shifts from detection to characterization with quantified uncertainty, supported by standardized metadata frameworks that enable measurement traceability and cross-project comparability.
\end{quote}

This hypothesis implies that current NDT capability may be sufficient for circular construction in principle, but that the limiting factor is metrological infrastructure rather than measurement technology. To evaluate this hypothesis, we address three research questions:

\begin{description}
\item[RQ1:] What is the current \emph{validated} capability of NDT methods to characterize the assessment tasks required for structural reuse (deterioration, strength, defects, moisture, geometry) across the four primary material classes (concrete, steel, timber, masonry)?

\item[RQ2:] What detection limits and measurement uncertainties have been documented, and do these enable derivation of design-code-compliant characteristic values?

\item[RQ3:] What constrains the transition from detection-focused to characterization-focused NDT, and what does this reveal about research and standardization priorities?
\end{description}

The scope encompasses established NDT methods with documented structural application, excluding laboratory-only emerging techniques. We focus on the four dominant structural material classes in existing building stock: reinforced concrete, structural steel, timber, and unreinforced masonry.

%=============================================================================
\section{Methodology}
\label{sec:methods}
%=============================================================================

\subsection{Systematic Search Strategy}
\label{sec:methods:search}

This review followed the Preferred Reporting Items for Systematic Reviews and Meta-Analyses (PRISMA) guidelines \citep{prisma2020}. Searches were conducted across Scopus, Web of Science, and Engineering Village using Boolean combinations:


% \begin{center}
% \begin{minipage}{0.85\linewidth}
% \begin{verbatim}
% ("non-destructive testing" OR "NDT" OR "NDE") AND ("concrete" OR "steel" OR "timber" OR "masonry" OR "building" OR "structure") AND
% ("assessment" OR "evaluation" OR "reuse" OR "circular" OR "characterization")
% \end{verbatim}
% \end{minipage}
% \end{center}

\begin{center}
\begin{minipage}{0.85\linewidth}
\begin{Verbatim}[breaklines=true, breakanywhere=true]
("non-destructive testing" OR "NDT" OR "NDE") AND ("concrete" OR "steel" OR "timber" OR "masonry" OR "building" OR "structure") AND
("assessment" OR "evaluation" OR "reuse" OR "circular" OR "characterization")
\end{Verbatim}
\end{minipage}
\end{center}


The search period covered January 2014 to December 2024, capturing the decade following major European standardization updates (EN~13791:2019, revised Eurocodes) and the emergence of circular economy policy frameworks.

From 612 initial records, 590 remained after duplicate removal. Title and abstract screening excluded 227 records not addressing structural materials or lacking quantitative performance data. Full-text assessment of 363 articles excluded studies reporting only laboratory specimens without field validation context, purely theoretical analyses, and non-peer-reviewed sources. The final synthesis incorporated 227 studies providing quantitative capability data for structural NDT applications.

\textcolor{red}{\textbf{Note to co-authors:} Zia---please verify these PRISMA numbers against your actual search results and update accordingly. The flow should match your Scopus/WoS exports. --- CV}\\

\textcolor{blue}{\textbf{Note to co-authors:} Christoph -- Should I include the PRISMA graph that we had in the previous paper?}


\PrismaSet{612}{37}{590}{590}{227}{363}{0}{227}{\textemdash}

\begin{figure}[t]
\centering
\resizebox{\columnwidth}{!}{%
\begin{tikzpicture}[node distance=1.2cm and 2.8cm]
% Identification
\node (id1) [prismaBox] {Records identified via databases\\(n=\nDB)};
\node (id2) [prismaBox, right=3.4cm of id1] {Additional records identified\\through other methods\\(n=\nOTH)};
\node (dedup) [prismaWide]
  at ($ (id1)!0.5!(id2) + (0,-1.8cm) $)
  {Records after duplicates removed\\(n=\nDEDUP)};
\draw[prismaArrow] (id1) -- (dedup);
\draw[prismaArrow] (id2) -- (dedup);

% Screening
\node (screen) [prismaWide, below=1.2cm of dedup] {Records screened (title/abstract)\\(n=\nSCREEN)};
\draw[prismaArrow] (dedup) -- (screen);
\node (excl1) [prismaBox, right=3.4cm of screen] {Records excluded\\(n=\nEXCLSCREEN)};
\draw[prismaArrow] (screen.east) -- (excl1.west);

% Eligibility
\node (full) [prismaWide, below=1.2cm of screen] {Full-text articles assessed for eligibility\\(n=\nFULL)};
\draw[prismaArrow] (screen) -- (full);
\node (excl2) [prismaBox, right=3.4cm of full] {Full-text articles excluded\\with reasons\\(n=\nEXCLFULL)};
\draw[prismaArrow] (full.east) -- (excl2.west);

% Inclusion
\node (qual) [prismaWide, below=1.2cm of full] {Studies included in qualitative synthesis\\(n=\nQUAL)};
\draw[prismaArrow] (full) -- (qual);
\node (quant) [prismaWide, below=1.0cm of qual] {Studies included in quantitative synthesis (meta-analysis)\\(n=\nQUANT)};
\draw[prismaArrow] (qual) -- (quant);
\end{tikzpicture}}
\caption{PRISMA-style screening flow showing identification (612 database + 37 other records), screening (590 after deduplication), and inclusion of 227 studies in the qualitative synthesis.}
\label{fig:prisma}
\end{figure}

\subsection{Capability Classification Framework}
\label{sec:methods:classification}

Each material-task-method combination was classified according to the criteria in \Cref{tab:capability_criteria}. This three-level scheme distinguishes between methods with full metrological characterization (Validated), methods demonstrating capability but lacking complete uncertainty budgets (Applicable), and methods with significant constraints limiting practical deployment (Limited).

\begin{table}[htbp]
\centering
\caption{Capability classification criteria}
\label{tab:capability_criteria}
\begin{tabular}{@{}lp{12cm}@{}}
\toprule
Level & Criteria \\
\midrule
\textbf{Validated} (++) & Detection limits quantified in field conditions; uncertainty budget documented per GUM \citep{gum2008}; multiple independent validation studies; standard operating procedures established \\
\textbf{Applicable} (+) & Detection demonstrated in representative conditions; uncertainty $>$20\% or incompletely characterized; limited field validation; correlation models available but not independently verified \\
\textbf{Limited} ($\circ$) & Method applicable in principle; significant practical constraints; insufficient validation data; uncertainty unknown or very large ($>$35\%) \\
\bottomrule
\end{tabular}
\end{table}

Classification required consensus assessment based on the totality of evidence for each combination. Where studies reported conflicting capability levels, we assigned the more conservative classification unless methodological differences explained the discrepancy.

\subsection{Metrological Framework}
\label{sec:methods:metrology}

The transformation from NDT measurement to design value follows a chain with uncertainty accumulating at each stage (\Cref{fig:chain}). Following the ISO Guide to the Expression of Uncertainty in Measurement (GUM), the combined standard uncertainty in a derived quantity comprises:
\begin{equation}
u_c(f_k) = \sqrt{u_\text{instr}^2 + u_\text{coupling}^2 + u_\text{material}^2 + u_\text{env}^2 + u_\text{operator}^2 + u_\text{corr}^2 + u_\text{model}^2}
\label{eq:uncertainty}
\end{equation}
where the components represent instrumental calibration ($u_\text{instr}$), sensor coupling variability ($u_\text{coupling}$), material heterogeneity within the element ($u_\text{material}$), environmental effects ($u_\text{env}$), operator-dependent variability ($u_\text{operator}$), correlation model uncertainty ($u_\text{corr}$), and interpretation model uncertainty ($u_\text{model}$).

For structural applications, the characteristic value must satisfy:
\begin{equation}
f_k = \bar{f} - k_n \cdot s
\label{eq:characteristic}
\end{equation}
where $\bar{f}$ is the mean estimate, $s$ the standard deviation incorporating all uncertainty sources, and $k_n$ the statistical factor depending on sample size and required fractile (typically 1.64 for 5\% fractile, adjusted for small samples per EN~1990 Annex~D).

A method achieves ``Validated'' status only when all uncertainty components in \Cref{eq:uncertainty} have been quantified and the combined uncertainty enables derivation of $f_k$ per \Cref{eq:characteristic} with acceptable precision for design purposes.

%=============================================================================
\section{Theoretical Framework}
\label{sec:theory}
%=============================================================================

\subsection{Assessment Requirements for Structural Reuse}
\label{sec:theory:requirements}

Structural reuse decisions require answers to five fundamental questions, each mapping to specific NDT assessment tasks:

\begin{enumerate}
\item \textbf{Geometry verification:} Are dimensions, cross-section, reinforcement layout, and connection details compatible with the intended application?
\item \textbf{Strength estimation:} What are the characteristic strength values, and are they sufficient for design requirements?
\item \textbf{Deterioration assessment:} What is the state of degradation mechanisms (corrosion, decay, chemical attack), and what is the residual capacity?
\item \textbf{Defect identification:} Are there cracks, voids, delaminations, or hidden damage from previous loading, fire, or impact?
\item \textbf{Moisture condition:} What is the current moisture content, and what is the susceptibility to moisture-related degradation?
\end{enumerate}

For reuse acceptance, questions 1--4 must yield quantitative answers with uncertainties compatible with partial factor design. Question 5 informs durability rather than immediate capacity but affects long-term reuse viability.

\subsection{Damage Mechanisms by Material Class}
\label{sec:theory:damage}

\Cref{tab:damage} summarizes the primary damage mechanisms requiring NDT characterization for each material class. The mechanisms differ fundamentally in their physical manifestation and the measurement principles capable of detecting them.

\begin{table}[htbp]
\centering
\caption{Primary damage mechanisms by structural material class}
\label{tab:damage}
\small
\begin{tabular}{@{}lp{12cm}@{}}
\toprule
Material & Primary Damage Mechanisms \\
\midrule
Reinforced Concrete & Chloride-induced corrosion; carbonation-induced corrosion; alkali-silica reaction (ASR); freeze-thaw damage; sulphate attack; fire damage; cracking (flexural, shear, settlement) \\
Structural Steel & Uniform corrosion; pitting corrosion; fatigue cracking; fire damage (residual deformation, microstructural change); connection degradation \\
Timber & Fungal decay (brown rot, white rot); insect attack; mechanical damage; fire charring; connection loosening; moisture cycling damage \\
Masonry & Mortar deterioration; unit degradation; efflorescence and salt crystallization; freeze-thaw; settlement cracking; wall tie corrosion; moisture penetration \\
\bottomrule
\end{tabular}
\end{table}

\subsection{The Measurement-to-Design-Value Chain}
\label{sec:theory:chain}

The path from raw NDT measurement to design resistance involves four transformations, each introducing uncertainty (\Cref{fig:chain}):

\begin{enumerate}
\item \textbf{Signal acquisition:} Raw sensor output (voltage, time-of-flight, amplitude) with instrumental uncertainty $u_\text{instr}$ and coupling variability $u_\text{coupling}$.

\item \textbf{Physical quantity derivation:} Transformation to engineering quantity (velocity, resistivity, potential) via calibration, introducing $u_\text{env}$ and $u_\text{operator}$.

\item \textbf{Material property correlation:} Conversion to material property (strength, section loss, moisture content) via empirical correlations, introducing $u_\text{corr}$.

\item \textbf{Design value derivation:} Statistical treatment to obtain characteristic value per EN~1990, incorporating $u_\text{material}$ and $u_\text{model}$.
\end{enumerate}

Most NDT literature reports performance at stages 1--2. The gap for circular construction lies in stages 3--4: validated correlation models and statistical frameworks for characteristic value derivation are scarce.

\subsection{Bridging Measurement and Design: The Role of Metadata Ontologies}
\label{sec:theory:ontology}

The measurement-to-design-value chain can only be traversed if measurement provenance is preserved throughout. A measurement without documented device identity, calibration status, environmental conditions, and spatial location cannot be reliably transformed into a design value---the uncertainty components in \Cref{eq:uncertainty} cannot be evaluated.

This requirement motivates standardized NDT metadata frameworks. The Circular Potential Information Model (CP-IM), developed within the REINCARNATE project, provides a machine-actionable schema capturing device identification, calibration status, test timing, spatial location, and measurement sequence \citep{morenotorres2021}. Such ontologies enable cross-project data comparability, uncertainty traceability, and integration with building information models---prerequisites for accumulating validated NDT datasets at scale.

Current adoption remains limited: fewer than 25\% of NDT datasets in the reviewed literature employ standardized metadata formats \citep{voelker2020}. This ``data tsunami'' problem---large measurement volumes with insufficient context---represents a significant barrier to the paradigm shift proposed in this review.

\textcolor{red}{\textbf{Note to co-authors:} The ontology (CP-IM) is an original REINCARNATE contribution and could be expanded to a full subsection with the metadata schema details (NDT\_Device\_Brand/Model/ID, LastCalibrationDate, Test\_TimeStart/End, Location, NumberInSeries) and discussion of FAIR data principles. Question: Should we expand this to ~1 page to properly position it as a project contribution, or keep it brief and reference Moreno Torres et al. (2021) for details? --- CV}

%=============================================================================
\section{Capability Analysis}
\label{sec:capability}
%=============================================================================

\subsection{Reinforced Concrete}
\label{sec:capability:concrete}

Reinforced concrete represents the most extensively studied material class, with the largest literature base addressing corrosion, strength, and defect assessment.

\subsubsection{Corrosion Assessment}

Corrosion evaluation combines electrochemical methods for activity detection with complementary techniques for consequence assessment (\Cref{tab:concrete_corrosion}).

%\begin{table}[htbp]
%\centering
%\caption{NDT capability for corrosion assessment in reinforced concrete}
%\label{tab:concrete_corrosion}
%\small
%\begin{tabular}{@{}p{2.5cm}p{2.5cm}p{2.5cm}p{4cm}c@{}}
%\toprule
%Method & Measurand & Detection/Accuracy & Key Limitations & Cap. \\
%\midrule
%HCP (absolute) & Corrosion probability & ASTM C876 thresholds & Carbonation shifts thresholds; moisture-dependent & $\circ$ \\
%HCP (gradient) & Corrosion localization & $\nabla E > 20$--30~mV/m & Per DGZfP B3; robust to carbonation & + \\
%HCP + Resistivity & Risk classification & Combined indices & Per RILEM TC 154-EMC; widely practiced & ++ \\
%Resistivity (Wenner) & Ionic transport & CoV 10--15\% & Geometry-dependent; saturation affects & + \\
%GPR & Rebar location, moisture & Cover $\pm$5~mm & Penetration limited in saturated concrete & ++ \\
%\bottomrule
%\end{tabular}
%\end{table}


\begin{table}[htbp]
\centering
\caption{NDT capability for corrosion assessment in reinforced concrete}
\label{tab:concrete_corrosion}
\small
\begin{tabularx}{\textwidth}{@{}p{2.5cm}p{2.5cm}p{3cm}X c@{}}
\toprule
Method & Measurand & Detection/Accuracy & Key Limitations & Cap. \\
\midrule
HCP (absolute) & Corrosion probability & ASTM C876 thresholds &
Carbonation shifts thresholds; moisture-dependent & $\circ$ \\

HCP (gradient) & Corrosion localization &
$\nabla E > 20$--30~mV/m &
Per DGZfP B3; robust to carbonation & + \\

HCP + Resistivity & Risk classification & Combined indices &
Per RILEM TC 154-EMC; widely practiced & ++ \\

Resistivity (Wenner) & Ionic transport & CoV 10--15\% &
Geometry-dependent; saturation affects & + \\

GPR & Rebar location, moisture & Cover $\pm$5~mm &
Penetration limited in saturated concrete & ++ \\
\bottomrule
\end{tabularx}
\end{table}


\textbf{Half-cell potential (HCP)} measurement is the most widely applied corrosion screening method. The traditional interpretation per ASTM~C876 uses absolute potential thresholds ($<$-350~mV vs.\ CSE indicates $>$90\% corrosion probability). However, carbonation shifts the reference potential approximately +150~mV, causing false negatives in carbonated structures \citep{rilem154emc}.

The gradient-based approach documented in DGZfP Merkblatt B3 addresses this limitation by identifying local anodes through spatial potential differences rather than absolute values \citep{dgzfp_b3}. Potential gradients exceeding 20--30~mV/m indicate localized corrosion activity regardless of carbonation state. Combined with resistivity mapping per RILEM TC 154-EMC, this approach achieves validated classification of corrosion risk zones \citep{rilem154emc}.

\textbf{Critical finding:} Combined HCP gradient and resistivity mapping achieves validated capability for corrosion \emph{detection and localization}, but \emph{quantification} of section loss requires destructive verification. No purely non-destructive method provides validated measurement of remaining rebar cross-section.

\subsubsection{Strength Estimation}

In-situ strength estimation combines indirect measurements with correlation models (\Cref{tab:concrete_strength}).

% \begin{table}[htbp]
% \centering
% \caption{NDT capability for strength estimation in reinforced concrete}
% \label{tab:concrete_strength}
% \small
% \begin{tabular}{@{}p{2.5cm}p{2cm}p{2.5cm}p{4cm}c@{}}
% \toprule
% Method & Measurand & Uncertainty & Key Limitations & Cap. \\
% \midrule
% SONREB & Compressive strength & $\pm$15--20\% (calibrated) & Requires $\geq$9 cores per EN~13791 Approach~B & + \\
% UPV alone & Quality indicator & $\pm$25--30\% & Poor strength correlation; aggregate/moisture effects & $\circ$ \\
% Rebound alone & Surface hardness & $\pm$25--35\% & Carbonation increases readings 10--40\% & $\circ$ \\
% Pull-out (CAPO) & Direct strength & $\pm$8--12\% & Semi-destructive; point measurement & ++ \\
% \bottomrule
% \end{tabular}
% \end{table}

\begin{table}[htbp]
\centering
\caption{NDT capability for strength estimation in reinforced concrete}
\label{tab:concrete_strength}
\small
\begin{tabularx}{\textwidth}{@{}p{2.cm}p{2.cm}p{4cm}X c@{}}
\toprule
Method & Measurand & Uncertainty & Key Limitations & Cap. \\
\midrule
SONREB & Compressive strength &
$\pm$15--20\% (calibrated) &
Requires $\geq$9 cores per EN~13791 Approach~B & + \\

UPV alone & Quality indicator &
$\pm$25--30\% &
Poor strength correlation; aggregate/moisture effects & $\circ$ \\

Rebound alone & Surface hardness &
$\pm$25--35\% &
Carbonation increases readings 10--40\% & $\circ$ \\

Pull-out (CAPO) & Direct strength &
$\pm$8--12\% &
Semi-destructive; point measurement & ++ \\
\bottomrule
\end{tabularx}
\end{table}


The \textbf{SONREB} combination (ultrasonic pulse velocity + rebound number) represents current best practice for non-destructive strength estimation. Multi-variate correlations reduce uncertainty compared to single methods \citep{breysse2012}. However, EN~13791 requires site-specific calibration: Approach~A demands minimum 3 cores, Approach~B requires 9 cores for reliable correlation \citep{en13791}.

This calibration requirement reflects the fundamental limitation: without cores from the specific structure, uncertainty increases to $\pm$25--30\%, insufficient for characteristic value derivation without prohibitive safety factors.

The \textbf{pull-out test} (CAPO-test, Lok-test) provides the most reliable strength assessment with $\pm$8--12\% uncertainty \citep{petersen1997}. Though semi-destructive (requiring repair of 50--75~mm diameter holes), it directly measures concrete tensile/shear capacity correlated to compressive strength. For reuse assessment where a limited number of point measurements suffices, pull-out testing achieves validated capability.

\textbf{Critical finding:} Purely non-destructive strength estimation cannot achieve the $<$15\% uncertainty required for reliable characteristic value derivation. Current practice requires either pull-out testing (semi-destructive) or core extraction for SONREB calibration. The research gap is not better sensors but validated statistical frameworks for combining sparse destructive data with dense NDT coverage.

\subsubsection{Defect Detection}

Defect detection employs acoustic and electromagnetic methods with complementary capabilities (\Cref{tab:concrete_defects}).

% \begin{table}[htbp]
% \centering
% \caption{NDT capability for defect detection in reinforced concrete}
% \label{tab:concrete_defects}
% \small
% \begin{tabular}{@{}p{2.5cm}p{2cm}p{2.5cm}p{4cm}c@{}}
% \toprule
% Method & Target & Detection Limit & Key Limitations & Cap. \\
% \midrule
% Impact-echo & Delamination, voids & $>$50~mm lateral; depth to $\sim$600~mm & Requires plate geometry & + \\
% Ultrasonic pulse-echo & Thickness, voids & Thickness $\pm$3\%; voids $>$30~mm & Dense reinforcement interference & + \\
% IR thermography & Near-surface defects & $>$100~mm lateral; depth $<$50~mm & Requires thermal gradient; weather-dependent & + \\
% GPR & Reinforcement, voids & Rebar spacing $>$50~mm; cover $<$300~mm & Penetration limited in wet concrete & ++ \\
% \bottomrule
% \end{tabular}
% \end{table}


\begin{table}[htbp]
\centering
\caption{NDT capability for defect detection in reinforced concrete}
\label{tab:concrete_defects}
\small
\begin{tabularx}{\textwidth}{@{}p{2.5cm}p{2.5cm}p{4cm}X c@{}}
\toprule
Method & Target & Detection Limit & Key Limitations & Cap. \\
\midrule
Impact-echo & Delamination, voids &
$>$50~mm lateral; depth to $\sim$600~mm &
Requires plate geometry & + \\

Ultrasonic pulse-echo & Thickness, voids &
Thickness $\pm$3\%; voids $>$30~mm &
Dense reinforcement interference & + \\

IR thermography & Near-surface defects &
$>$100~mm lateral; depth $<$50~mm &
Requires thermal gradient; weather-dependent & + \\

GPR & Reinforcement, voids &
Rebar spacing $>$50~mm; cover $<$300~mm &
Penetration limited in wet concrete & ++ \\
\bottomrule
\end{tabularx}
\end{table}


\textbf{Ground-penetrating radar (GPR)} achieves validated capability for reinforcement mapping and cover depth measurement in typical conditions. The SHRP2 comparative studies demonstrated that GPR provides reliable rebar detection with cover accuracy of $\pm$5~mm for depths below 150~mm \citep{shrp2_2013}. Penetration in saturated concrete reduces to $<$100~mm, limiting applicability for moisture-affected structures.

\textbf{Impact-echo} and \textbf{ultrasonic pulse-echo} provide complementary delamination detection. The minimum detectable defect size is approximately $\lambda/3$ to $\lambda/2$, corresponding to 20--100~mm for typical frequencies \citep{sansalone1997}. Combined methods outperform single-method inspection: SHRP2 studies showed that multi-method approaches significantly improved detection reliability for bridge deck assessment \citep{shrp2_2013}.

\subsection{Structural Steel}
\label{sec:capability:steel}

Structural steel assessment benefits from mature NDT methods developed for manufacturing and petrochemical sectors, though building-specific applications present distinct challenges.

\subsubsection{Section Loss and Corrosion}

Steel corrosion assessment employs established industrial techniques (\Cref{tab:steel_corrosion}).

% \begin{table}[htbp]
% \centering
% \caption{NDT capability for corrosion/section loss in structural steel}
% \label{tab:steel_corrosion}
% \small
% \begin{tabular}{@{}p{2.5cm}p{2cm}p{2.5cm}p{4cm}c@{}}
% \toprule
% Method & Measurand & Detection Limit & Key Limitations & Cap. \\
% \midrule
% UT thickness & Section loss & $\pm$0.1~mm & Requires surface preparation & ++ \\
% MFL & Section loss, pitting & $>$10\% wall loss & Complex geometries difficult & ++ \\
% Pulsed EC & Wall loss through coating & $\pm$5\% through $<$10~mm coating & Lower resolution than UT & + \\
% Visual + measurement & Gross section loss & Surface visible only & Requires coating removal & ++ \\
% \bottomrule
% \end{tabular}
% \end{table}


\begin{table}[htbp]
\centering
\caption{NDT capability for corrosion/section loss in structural steel}
\label{tab:steel_corrosion}
\small
\begin{tabularx}{\textwidth}{@{}p{2.2cm}p{2.7cm}p{4cm}X c@{}}
\toprule
Method & Measurand & Detection Limit & Key Limitations & Cap. \\
\midrule
UT thickness & Section loss &
$\pm$0.1~mm &
Requires surface preparation & ++ \\

MFL & Section loss, pitting &
$>$10\% wall loss &
Complex geometries difficult & ++ \\

Pulsed EC & Wall loss through coating &
$\pm$5\% through $<$10~mm coating &
Lower resolution than UT & + \\

Visual + measurement & Gross section loss &
Surface visible only &
Requires coating removal & ++ \\
\bottomrule
\end{tabularx}
\end{table}


\textbf{Ultrasonic thickness measurement} represents a mature, validated technology with $\pm$0.1~mm accuracy achievable with proper surface preparation \citep{krautkramer2013}. The primary challenge in building applications is access to connection regions and concealed members.

\textbf{Magnetic flux leakage (MFL)} provides rapid screening for section loss exceeding 10\% of wall thickness, with validated detection capability in accessible plate and tubular members.

\subsubsection{Fatigue and Crack Detection}

Crack detection benefits from extensive industrial validation (\Cref{tab:steel_cracks}).

% \begin{table}[htbp]
% \centering
% \caption{NDT capability for crack detection in structural steel}
% \label{tab:steel_cracks}
% \small
% \begin{tabular}{@{}p{2.5cm}p{2cm}p{2.5cm}p{4cm}c@{}}
% \toprule
% Method & Measurand & Detection Limit & Key Limitations & Cap. \\
% \midrule
% Magnetic particle (MT) & Surface cracks & Length $>$2~mm & Surface preparation required & ++ \\
% Dye penetrant (PT) & Surface cracks & Similar to MT & Clean, dry surface required & ++ \\
% ACFM & Surface cracks & Depth $>$1~mm & Through coatings $<$5~mm & + \\
% Phased array UT & Subsurface cracks & Height $>$1~mm & Geometry-dependent & ++ \\
% TOFD & Crack sizing & Height $\pm$1~mm & Near-surface dead zone & ++ \\
% \bottomrule
% \end{tabular}
% \end{table}



\begin{table}[htbp]
\centering
\caption{NDT capability for crack detection in structural steel}
\label{tab:steel_cracks}
\small
\begin{tabularx}{\textwidth}{@{}p{4cm}p{2.5cm}p{2.5cm}X c@{}}
\toprule
Method & Measurand & Detection Limit & Key Limitations & Cap. \\
\midrule
Magnetic particle (MT) & Surface cracks &
Length $>$2~mm &
Surface preparation required & ++ \\

Dye penetrant (PT) & Surface cracks &
Similar to MT &
Clean, dry surface required & ++ \\

ACFM & Surface cracks &
Depth $>$1~mm &
Through coatings $<$5~mm & + \\

Phased array UT & Subsurface cracks &
Height $>$1~mm &
Geometry-dependent & ++ \\

TOFD & Crack sizing &
Height $\pm$1~mm &
Near-surface dead zone & ++ \\
\bottomrule
\end{tabularx}
\end{table}


\textbf{Magnetic particle} and \textbf{penetrant testing} achieve validated surface crack detection per ASME and ISO standards with documented POD curves \citep{asme2019}. \textbf{Phased array UT} and \textbf{TOFD} provide volumetric detection with crack height accuracy of $\pm$1~mm for defects exceeding 3~mm.

\subsubsection{Fire Damage Assessment}

Fire damage assessment lacks validated NDT methods for property confirmation (\Cref{tab:steel_fire}).

% \begin{table}[htbp]
% \centering
% \caption{NDT capability for fire damage assessment in structural steel}
% \label{tab:steel_fire}
% \small
% \begin{tabular}{@{}p{2.5cm}p{2cm}p{2.5cm}p{4cm}c@{}}
% \toprule
% Method & Indicator & Capability & Key Limitations & Cap. \\
% \midrule
% Hardness testing & Yield strength proxy & Qualitative screening & Surface only; cold work affects & + \\
% Residual deformation & Temperature indicator & $>$600°C exposure & No property data & + \\
% Metallographic replica & Microstructure & Laboratory confirmation & Interpretation expertise required & $\circ$ \\
% \bottomrule
% \end{tabular}
% \end{table}

\begin{table}[htbp]
\centering
\caption{NDT capability for fire damage assessment in structural steel}
\label{tab:steel_fire}
\small
\begin{tabularx}{\textwidth}{@{}p{3.3cm}p{2.5cm}p{3cm}X c@{}}
\toprule
Method & Indicator & Capability & Key Limitations & Cap. \\
\midrule
Hardness testing & Yield strength proxy &
Qualitative screening &
Surface only; cold work affects & + \\

Residual deformation & Temperature indicator &
$>$600°C exposure &
No property data & + \\

Metallographic replica & Microstructure &
Laboratory confirmation &
Interpretation expertise required & $\circ$ \\
\bottomrule
\end{tabularx}
\end{table}




\textbf{Critical finding:} No validated NDT method confirms microstructural changes affecting mechanical properties after fire. Hardness testing and residual deformation provide qualitative indicators, but property confirmation requires destructive sampling \citep{outinen2004}.

\subsection{Timber}
\label{sec:capability:timber}

Timber assessment presents unique challenges due to material anisotropy, biological degradation mechanisms, and difficulty detecting internal decay without surface indication.

\subsubsection{Decay Detection}

Decay assessment addresses the primary reuse concern: internal degradation that may not manifest at the surface (\Cref{tab:timber_decay}).

% \begin{table}[htbp]
% \centering
% \caption{NDT capability for decay assessment in timber}
% \label{tab:timber_decay}
% \small
% \begin{tabular}{@{}p{2.5cm}p{2cm}p{2.5cm}p{4cm}c@{}}
% \toprule
% Method & Measurand & Detection Limit & Key Limitations & Cap. \\
% \midrule
% Resistance drilling & Density profile & $>$20\% density loss & Point measurement; semi-destructive & + \\
% Stress wave timing & Internal decay & $>$30\% section loss & Moisture and anisotropy effects & $\circ$ \\
% Ultrasonic tomography & Decay mapping & Qualitative imaging & Multiple access points; interpretation complex & $\circ$ \\
% Visual + probing & Surface decay & Surface accessible only & Concealed decay undetected & + \\
% \bottomrule
% \end{tabular}
% \end{table}


\begin{table}[htbp]
\centering
\caption{NDT capability for decay assessment in timber}
\label{tab:timber_decay}
\small
\begin{tabularx}{\textwidth}{@{}p{3cm}p{2.5cm}p{3.3cm}X c@{}}
\toprule
Method & Measurand & Detection Limit & Key Limitations & Cap. \\
\midrule
Resistance drilling & Density profile &
$>$20\% density loss &
Point measurement; semi-destructive & + \\

Stress wave timing & Internal decay &
$>$30\% section loss &
Moisture and anisotropy effects & $\circ$ \\

Ultrasonic tomography & Decay mapping &
Qualitative imaging &
Multiple access points; interpretation complex & $\circ$ \\

Visual + probing & Surface decay &
Surface accessible only &
Concealed decay undetected & + \\
\bottomrule
\end{tabularx}
\end{table}


\textbf{Resistance drilling} (Resistograph) provides the most reliable decay assessment, achieving approximately 80\% detection for density loss exceeding 20\% \citep{rinn1996}. The method is semi-destructive (3~mm diameter penetration) and provides only point measurements, requiring systematic sampling for comprehensive assessment.

\textbf{Stress wave} methods detect internal decay when section loss exceeds approximately 30\%, but interpretation is complicated by anisotropy and moisture effects \citep{ross2015}. At this detection threshold, decay is already advanced---too late for many reuse scenarios.

\textbf{Critical finding:} No method achieves validated detection of \emph{incipient} decay ($<$20\% density loss). The practical approach combines visual inspection with systematic resistance drilling at suspect locations. For heritage structures where drilling may be unacceptable, stress wave screening can identify areas of concern but cannot exclude early-stage decay.

\subsubsection{Strength Estimation}

Timber strength estimation relies on indicators correlated to grade (\Cref{tab:timber_strength}).

% \begin{table}[htbp]
% \centering
% \caption{NDT capability for strength estimation in timber}
% \label{tab:timber_strength}
% \small
% \begin{tabular}{@{}p{2.5cm}p{2cm}p{2.5cm}p{4cm}c@{}}
% \toprule
% Method & Measurand & Uncertainty & Key Limitations & Cap. \\
% \midrule
% Stress wave velocity & Dynamic MOE & $\pm$20--30\% for strength & Poor bending strength correlation & $\circ$ \\
% Resistance drilling & Density correlation & $\pm$25--35\% for strength & Point measurement; species-dependent & $\circ$ \\
% Visual grading & Defect-based & Grade-dependent & Trained grader required; concealed defects missed & + \\
% \bottomrule
% \end{tabular}
% \end{table}

\begin{table}[htbp]
\centering
\caption{NDT capability for strength estimation in timber}
\label{tab:timber_strength}
\small
\begin{tabularx}{\textwidth}{@{}p{3cm}p{2.5cm}p{3.5cm}X c@{}}
\toprule
Method & Measurand & Uncertainty & Key Limitations & Cap. \\
\midrule
Stress wave velocity & Dynamic MOE &
$\pm$20--30\% for strength &
Poor bending strength correlation & $\circ$ \\

Resistance drilling & Density correlation &
$\pm$25--35\% for strength &
Point measurement; species-dependent & $\circ$ \\

Visual grading & Defect-based &
Grade-dependent &
Trained grader required; concealed defects missed & + \\
\bottomrule
\end{tabularx}
\end{table}


Visual grading per EN~14081 remains the reference method for in-situ timber, correlating visible defects (knots, slope of grain, checks) to strength classes \citep{en14081}. Machine grading methods achieve tighter correlations but require controlled conditions not available in-situ.

\textbf{Critical finding:} No NDT method achieves strength estimation uncertainty below $\pm$20\% for in-situ timber. The combination of visual grading with stress wave measurement represents current best practice but provides only coarse screening.

\subsubsection{Moisture Assessment}

Moisture measurement is the most validated timber NDT capability (\Cref{tab:timber_moisture}).

% \begin{table}[htbp]
% \centering
% \caption{NDT capability for moisture assessment in timber}
% \label{tab:timber_moisture}
% \small
% \begin{tabular}{@{}p{2.5cm}p{2cm}p{2.5cm}p{4cm}c@{}}
% \toprule
% Method & Measurand & Accuracy & Key Limitations & Cap. \\
% \midrule
% Resistance meter & MC (surface) & $\pm$1--2\% below FSP & Above 30\% unreliable & ++ \\
% Dielectric meter & MC (deeper) & $\pm$2--3\% & Species calibration needed & + \\
% Oven drying & Actual MC & $\pm$0.1\% & Destructive reference & ++ \\
% \bottomrule
% \end{tabular}
% \end{table}


\begin{table}[htbp]
\centering
\caption{NDT capability for moisture assessment in timber}
\label{tab:timber_moisture}
\small
\begin{tabularx}{\textwidth}{@{}p{3cm}p{2.5cm}p{3cm}X c@{}}
\toprule
Method & Measurand & Accuracy & Key Limitations & Cap. \\
\midrule
Resistance meter & MC (surface) &
$\pm$1--2\% below FSP &
Above 30\% unreliable & ++ \\

Dielectric meter & MC (deeper) &
$\pm$2--3\% &
Species calibration needed & + \\

Oven drying & Actual MC &
$\pm$0.1\% &
Destructive reference & ++ \\
\bottomrule
\end{tabularx}
\end{table}


\textbf{Resistance-type moisture meters} achieve $\pm$1--2\% accuracy below the fiber saturation point (approximately 28\% MC) \citep{skaar1988}. Above FSP, accuracy degrades significantly---an unfortunate inversion since decay risk increases precisely in this moisture range.

\subsection{Masonry}
\label{sec:capability:masonry}

Masonry assessment is complicated by the composite nature of the material system (units + mortar), variability in historic construction, and limited penetration of many NDT methods.

\subsubsection{Strength and Deformability}

Masonry strength assessment has limited non-destructive options (\Cref{tab:masonry_strength}).

% \begin{table}[htbp]
% \centering
% \caption{NDT capability for strength assessment in masonry}
% \label{tab:masonry_strength}
% \small
% \begin{tabular}{@{}p{2.5cm}p{2cm}p{2.5cm}p{4cm}c@{}}
% \toprule
% Method & Measurand & Uncertainty & Key Limitations & Cap. \\
% \midrule
% Flat-jack (single) & In-situ stress & $\pm$15--20\% & Semi-destructive; point measurement & + \\
% Flat-jack (double) & Stress-strain & Deformability curve & More invasive; stress concentration & + \\
% Core extraction & Direct strength & Reference method & Highly destructive & ++ \\
% Schmidt hammer & Mortar quality & Qualitative only & No masonry strength correlation & $\circ$ \\
% UPV & Quality indicator & No strength correlation & Joint interfaces affect results & $\circ$ \\
% \bottomrule
% \end{tabular}
% \end{table}


\begin{table}[htbp]
\centering
\caption{NDT capability for strength assessment in masonry}
\label{tab:masonry_strength}
\small
\begin{tabularx}{\textwidth}{@{}p{3cm}p{2.5cm}p{3cm}X c@{}}
\toprule
Method & Measurand & Uncertainty & Key Limitations & Cap. \\
\midrule
Flat-jack (single) & In-situ stress &
$\pm$15--20\% &
Semi-destructive; point measurement & + \\

Flat-jack (double) & Stress-strain &
Deformability curve &
More invasive; stress concentration & + \\

Core extraction & Direct strength &
Reference method &
Highly destructive & ++ \\

Schmidt hammer & Mortar quality &
Qualitative only &
No masonry strength correlation & $\circ$ \\

UPV & Quality indicator &
No strength correlation &
Joint interfaces affect results & $\circ$ \\
\bottomrule
\end{tabularx}
\end{table}


The \textbf{flat-jack} technique---inserting a thin hydraulic jack into a cut mortar joint---provides the most reliable in-situ assessment. Single flat-jack measures in-situ stress; double flat-jack provides stress-strain relationships for deformability estimation \citep{astm_c1196,astm_c1197}. The method has extensive validation, particularly in Italian practice for historic masonry per CNR-DT 200 and RILEM TC 127-MS recommendations \citep{binda2000}.

Uncertainty of $\pm$15--20\% for compressive strength estimation is achievable with proper execution \citep{gregorczyk2000}. While semi-destructive (requiring mortar joint cutting and repair), flat-jack testing achieves the closest approach to validated in-situ strength characterization for masonry.

Surface methods (\textbf{Schmidt hammer}, \textbf{UPV}) show poor correlation to masonry assembly strength due to composite unit-mortar behavior. Rebound readings on mortar joints indicate mortar quality but cannot predict assembly strength.

\textbf{Critical finding:} Flat-jack testing achieves applicable (but not fully validated) capability for masonry strength estimation. The method is semi-destructive but represents the only realistic path to in-situ strength characterization without core extraction. Full validation requires additional round-robin studies documenting reproducibility across operators and masonry types.

\subsubsection{Defect Detection}

Defect detection in masonry employs imaging methods with limited quantitative capability (\Cref{tab:masonry_defects}).

% \begin{table}[htbp]
% \centering
% \caption{NDT capability for defect detection in masonry}
% \label{tab:masonry_defects}
% \small
% \begin{tabular}{@{}p{2.5cm}p{2cm}p{2.5cm}p{4cm}c@{}}
% \toprule
% Method & Target & Detection Limit & Key Limitations & Cap. \\
% \midrule
% IR thermography & Delamination, moisture & Qualitative & Weather-dependent; surface only & + \\
% GPR & Voids, thickness & Voids $>$50~mm & Multi-wythe walls complex & $\circ$ \\
% Sonic testing & Internal structure & Qualitative imaging & Low resolution & $\circ$ \\
% Borescope & Cavity inspection & Visual access required & Point inspection only & + \\
% \bottomrule
% \end{tabular}
% \end{table}

\begin{table}[htbp]
\centering
\caption{NDT capability for defect detection in masonry}
\label{tab:masonry_defects}
\small
\begin{tabularx}{\textwidth}{@{}p{2.7cm}p{2.7cm}p{3.9cm}X c@{}}
\toprule
Method & Target & Detection Limit & Key Limitations & Cap. \\
\midrule
IR thermography & Delamination, moisture &
Qualitative &
Weather-dependent; surface only & + \\

GPR & Voids, thickness &
Voids $>$50~mm &
Multi-wythe walls complex & $\circ$ \\

Sonic testing & Internal structure &
Qualitative imaging &
Low resolution & $\circ$ \\

Borescope & Cavity inspection &
Visual access required &
Point inspection only & + \\
\bottomrule
\end{tabularx}
\end{table}




\textbf{Infrared thermography} provides useful qualitative information on delamination, voids, and moisture patterns but cannot quantify internal structure. \textbf{GPR} struggles with multi-wythe walls where mortar joints create extensive signal scattering.

\textbf{Critical finding:} Masonry defect detection remains largely qualitative. The heterogeneous structure, variable mortar joints, and multi-wythe construction limit the effectiveness of methods developed for more homogeneous materials.

%=============================================================================
\section{Synthesis: The Capability Matrix}
\label{sec:synthesis}
%=============================================================================

\subsection{Quantitative Coverage Analysis}
\label{sec:synthesis:coverage}

The consolidated Capability Matrix (\Cref{tab:matrix}) summarizes NDT capabilities across 47 material-task-method combinations. This synthesis constitutes the primary empirical finding of this review.

\begin{table}[htb]
\centering
\caption{Capability Matrix: NDT methods for circular construction assessment}
\label{tab:matrix}
\small
\begin{tabular}{@{}l cccc cccc cccc@{}}
\toprule
& \multicolumn{4}{c}{Deterioration} & \multicolumn{4}{c}{Strength} & \multicolumn{4}{c}{Defects} \\
\cmidrule(lr){2-5} \cmidrule(lr){6-9} \cmidrule(lr){10-13}
Method & RC & St & Ti & Ma & RC & St & Ti & Ma & RC & St & Ti & Ma \\
\midrule
\textit{Acoustic} \\
UPV & $\circ$ & -- & $\circ$ & $\circ$ & $\circ$ & -- & $\circ$ & $\circ$ & + & -- & $\circ$ & $\circ$ \\
Impact-echo & -- & -- & -- & $\circ$ & -- & -- & -- & -- & + & -- & -- & $\circ$ \\
UT/PAUT & -- & ++ & -- & -- & -- & -- & -- & -- & + & ++ & -- & -- \\
\textit{Electromagnetic} \\
GPR & + & -- & -- & $\circ$ & -- & -- & -- & -- & ++ & -- & -- & $\circ$ \\
Eddy current & -- & + & -- & -- & -- & -- & -- & -- & -- & + & -- & -- \\
MFL & -- & ++ & -- & -- & -- & -- & -- & -- & -- & ++ & -- & -- \\
\textit{Electrochemical} \\
HCP (gradient) & + & -- & -- & -- & -- & -- & -- & -- & -- & -- & -- & -- \\
HCP + Resistivity & ++ & -- & -- & -- & -- & -- & -- & -- & -- & -- & -- & -- \\
\textit{Mechanical} \\
SONREB & -- & -- & -- & -- & + & -- & -- & -- & -- & -- & -- & -- \\
Pull-out/CAPO & -- & -- & -- & -- & ++ & -- & -- & -- & -- & -- & -- & -- \\
Flat-jack & -- & -- & -- & -- & -- & -- & -- & + & -- & -- & -- & -- \\
Resistance drilling & -- & -- & + & -- & -- & -- & $\circ$ & -- & -- & -- & + & -- \\
\textit{Surface} \\
MT/PT & -- & ++ & -- & -- & -- & -- & -- & -- & -- & ++ & -- & -- \\
Visual grading & + & + & + & + & -- & -- & + & -- & + & + & + & + \\
\textit{Thermal} \\
IR thermography & -- & -- & -- & -- & -- & -- & -- & -- & + & -- & -- & + \\
\textit{Moisture} \\
Resistance meter & -- & -- & -- & -- & -- & -- & -- & -- & -- & -- & -- & -- \\
\bottomrule
\multicolumn{13}{l}{\footnotesize RC=Reinforced Concrete; St=Steel; Ti=Timber; Ma=Masonry} \\
\multicolumn{13}{l}{\footnotesize ++=Validated; +=Applicable; $\circ$=Limited; --=Not applicable}
\end{tabular}
\end{table}





\Cref{tab:coverage} quantifies capability coverage by material class. The analysis reveals that only 21\% of material-task-method combinations achieve validated status with field-confirmed detection limits and documented uncertainty budgets.

\textcolor{red}{\textbf{Note to co-authors:} The 21\% figure and all Table 16 classifications need verification against Zia's literature database. Please cross-check each cell before submission. --- CV}

\begin{table}[H]
\centering
\caption{Capability coverage by material class}
\label{tab:coverage}
\small
\begin{tabular}{@{}lrrrrr@{}}
\toprule
Material & Total & Validated & Applicable & Limited & Not Applicable \\
\midrule
Reinforced Concrete & 15 & 3 (20\%) & 7 (47\%) & 4 (27\%) & 1 (7\%) \\
Structural Steel & 12 & 6 (50\%) & 4 (33\%) & 0 (0\%) & 2 (17\%) \\
Timber & 10 & 1 (10\%) & 3 (30\%) & 5 (50\%) & 1 (10\%) \\
Masonry & 10 & 0 (0\%) & 5 (50\%) & 4 (40\%) & 1 (10\%) \\
\midrule
\textbf{Total} & \textbf{47} & \textbf{10 (21\%)} & \textbf{19 (40\%)} & \textbf{13 (28\%)} & \textbf{5 (11\%)} \\
\bottomrule
\end{tabular}
\end{table}

\subsection{Material-Specific Patterns}
\label{sec:synthesis:materials}

\textbf{Structural steel achieves 50\% validated coverage}, reflecting technology transfer from manufacturing and petrochemical industries where POD studies and field validation are standard practice. Methods including UT thickness measurement, MT/PT for crack detection, and MFL for section loss have extensive industrial deployment with documented reliability.

\textbf{Reinforced concrete achieves 20\% validated coverage}, concentrated in geometry verification (GPR for rebar mapping) and combined electrochemical methods for corrosion detection. The extensive ``applicable'' category (47\%) represents methods that can detect or measure targets but lack the uncertainty characterization required for design value derivation.

\textbf{Timber achieves only 10\% validated coverage}---limited to moisture measurement below fiber saturation point. Critically, no method achieves validated capability for the primary reuse concern: internal decay detection.

\textbf{Masonry achieves 0\% validated coverage}. While flat-jack testing is classified as ``applicable'' for strength estimation, it lacks the full validation (multi-operator round-robin studies, documented reproducibility) required for ``validated'' status. This represents the most critical gap given that masonry structures constitute a large fraction of European building stock eligible for selective deconstruction.

\subsection{The Binding Constraint: Uncertainty Propagation}
\label{sec:synthesis:constraint}

The quantitative analysis reveals that the fundamental limitation is not detection capability but \emph{uncertainty propagation} from measurement to design value.

Consider concrete strength estimation via SONREB. Laboratory studies report coefficient of variation (CoV) of 10--15\%. Field studies report 15--25\%. This factor-of-two increase reflects unquantified field effects: surface carbonation, moisture variation, aggregate heterogeneity, reinforcement interference. Without systematic field validation, we cannot determine whether a given measurement's uncertainty is 15\% or 25\%---an unacceptable ambiguity for characteristic value derivation.

The gap between ``applicable'' and ``validated'' is not primarily about improved instrumentation. It reflects:
\begin{enumerate}
\item Incomplete uncertainty budgets: field variability, operator effects, and correlation model uncertainty remain unquantified for most methods
\item Lack of systematic field validation: laboratory-derived correlations have not been verified at scale in real structures
\item Absent characteristic value derivation frameworks: no validated methodology transforms NDT measurements into EN~1990-compliant characteristic values with appropriate partial factors
\end{enumerate}

\textbf{Core finding:} The constraint preventing NDT-enabled circular construction is metrological rather than technological. Existing methods can detect deterioration, estimate properties, and identify defects. What is missing is the metrological infrastructure---validated uncertainty budgets, documented detection limits, standardized correlation models---that would enable design value derivation with quantified reliability.

%=============================================================================
\section{Discussion}
\label{sec:discussion}
%=============================================================================

\subsection{Enabling the Paradigm Shift}
\label{sec:discussion:enabling}

The transition from detection-focused to characterization-focused NDT requires three enabling developments:

\textbf{Validated correlation frameworks.} Current correlation models (e.g., SONREB to strength) derive from laboratory studies or limited field campaigns. Enabling design value derivation requires correlations validated across representative structures, concrete types, and environmental conditions. This is a standardization challenge, not a research frontier: the technical approach is established, but systematic multi-site validation campaigns have not been conducted.

\textbf{Uncertainty budgets per GUM.} Complete uncertainty budgets following ISO GUM require quantification of all components in \Cref{eq:uncertainty}. For most NDT methods, only instrumental uncertainty ($u_\text{instr}$) is documented. The dominant components for in-situ assessment---material heterogeneity ($u_\text{material}$), operator variability ($u_\text{operator}$), and correlation model uncertainty ($u_\text{corr}$)---remain uncharacterized. Round-robin studies with multiple operators assessing the same structures would provide this data.

\textbf{Standardized metadata for traceability.} The measurement-to-design-value chain can only be traversed if measurement provenance is documented. The CP-IM ontology (\Cref{sec:theory:ontology}) provides a framework, but adoption remains limited. Integration of NDT metadata requirements into assessment standards (EN~13791, EN~17121) would accelerate adoption.

\subsection{Implications for Practice}
\label{sec:discussion:practice}

The findings have direct implications for stakeholders pursuing circular construction:

\textbf{For building owners and developers:} Current claims of ``non-destructive reuse assessment'' require critical evaluation. NDT methods provide valuable screening and spatial mapping, but characteristic value derivation for most properties still requires destructive sampling. Economic models for component reuse must account for validation testing costs, not just NDT measurement costs.

\textbf{For structural engineers:} Reuse assessment demands explicit uncertainty quantification---implicit ``engineering judgment'' is insufficient for design value derivation. Until validated frameworks exist, conservative safety factors or destructive sampling remain necessary. Engineers should demand validated uncertainty data from NDT service providers.

\textbf{For standards bodies:} The path to enabling circular construction lies through validation of existing methods rather than development of new technologies. Priority actions include: (i) commissioning multi-site validation campaigns for ``applicable'' methods; (ii) developing uncertainty quantification guidance integrated with EN~1990 partial factor methodology; (iii) incorporating metadata requirements into assessment standards.

\subsection{The Path Forward: Validation over Innovation}
\label{sec:discussion:path}

The review findings redirect research priorities. The constraint is not technological---we have sensors capable of detecting and estimating the required properties. The constraint is metrological: we lack the validation infrastructure to transform measurements into design values.

The most impactful research agenda is therefore:
\begin{enumerate}
\item \textbf{Field validation campaigns:} Systematic studies comparing NDT-derived values to destructive reference measurements across representative structures, documenting all uncertainty components.
\item \textbf{Operator variability studies:} Round-robin exercises quantifying $u_\text{operator}$ for key methods, establishing realistic field CoV.
\item \textbf{Correlation model validation:} Independent verification of published correlations (SONREB, stress wave, resistance drilling) across material types and conditions.
\item \textbf{Statistical framework development:} Methodology for combining sparse destructive data with dense NDT coverage to derive characteristic values per EN~1990.
\item \textbf{Metadata standard adoption:} Integration of CP-IM or equivalent ontologies into assessment workflows, enabling cross-project data accumulation.
\end{enumerate}

This agenda prioritizes enabling the 40\% of ``applicable'' methods to transition to ``validated'' status through metrological rigor, rather than developing new sensors that would face the same validation gap.

%=============================================================================
\section{Conclusions}
\label{sec:conclusions}
%=============================================================================

This systematic review evaluated the capability of non-destructive testing methods to support circular construction through reliable assessment of existing structural elements. The analysis synthesized 227 studies across four material classes and constructed a Capability Matrix classifying 47 material-task-method combinations by validation status.

\subsection*{Hypothesis Evaluation}

The central hypothesis proposed that NDT can bridge the gap between unknown in-situ conditions and design-code-compliant characteristic values, \emph{provided that} the paradigm shifts from detection to characterization with quantified uncertainty, supported by standardized metadata frameworks.

\textbf{The hypothesis is partially confirmed.} The technological capability exists: NDT methods can detect deterioration, estimate material properties, identify defects, and map geometry across all four material classes. However, the metrological infrastructure required to transform these measurements into design values is largely absent. Only 21\% of material-task-method combinations achieve validated status with documented uncertainty budgets enabling characteristic value derivation.

The hypothesis correctly identifies the paradigm shift as the critical enabler. Current practice remains detection-focused, answering ``is there corrosion?'' rather than ``what is the remaining section with $\pm$X\% uncertainty?'' The ontology component of the hypothesis is also confirmed: measurement provenance documentation is essential for uncertainty propagation, yet fewer than 25\% of datasets employ standardized metadata.

\subsection*{Research Question Answers}

\textbf{RQ1: What is the validated capability of NDT methods for structural reuse assessment?}

Validated capability is material-dependent: structural steel achieves 50\% coverage through technology transfer from industrial NDT; reinforced concrete achieves 20\%, concentrated in geometry verification and combined electrochemical corrosion detection; timber achieves 10\%, limited to moisture measurement; masonry achieves 0\% validated coverage. The majority of method applications are classified as ``applicable'' (40\%)---capable of detection but lacking validated uncertainty quantification.

\textbf{RQ2: What detection limits and uncertainties enable design value derivation?}

Validated detection limits exist for a small subset: UT thickness in steel ($\pm$0.1~mm), GPR cover in concrete ($\pm$5~mm), MT/PT crack detection ($>$2~mm), resistance moisture measurement in timber ($\pm$1--2\% below FSP). For strength estimation---the most critical parameter for reuse decisions---no purely non-destructive method achieves the $<$15\% uncertainty required for characteristic value derivation without site-specific calibration requiring destructive samples.

\textbf{RQ3: What constrains the transition to characterization-focused NDT?}

The binding constraint is metrological, not technological. Three specific gaps prevent progress: (i) incomplete uncertainty budgets with unquantified field variability, operator effects, and correlation model uncertainty; (ii) absence of systematic field validation comparing NDT-derived values to destructive references at scale; (iii) no established framework for deriving EN~1990-compliant characteristic values from NDT measurements with appropriate partial factors.

\subsection*{Outlook}

Enabling NDT-based circular construction requires investment in metrological infrastructure rather than sensor development. The research priorities are:

\begin{enumerate}
\item \textbf{Short-term (1--3 years):} Commissioning of multi-site validation campaigns for methods currently classified as ``applicable,'' particularly SONREB for concrete strength, combined HCP/resistivity for corrosion assessment, and flat-jack for masonry strength. These campaigns should document all GUM uncertainty components and enable transition to ``validated'' status.

\item \textbf{Medium-term (3--5 years):} Development of statistical frameworks for combining sparse destructive sampling with dense NDT coverage, enabling characteristic value derivation with quantified reliability. Integration of these frameworks into revised EN~13791 and development of equivalent standards for steel, timber, and masonry.

\item \textbf{Long-term (5--10 years):} Routine adoption of standardized metadata ontologies (CP-IM or equivalent) enabling cross-project data accumulation. As validated datasets grow, development of structure-type-specific correlations with reduced calibration requirements, eventually enabling truly non-destructive reuse assessment for common structural configurations.
\end{enumerate}

The opportunity is clear: the technological capability exists, and the path to validation is well-defined. The 40\% of methods classified as ``applicable'' represent low-hanging fruit---requiring validation rather than invention. Systematic investment in metrological infrastructure could transform NDT from a screening tool into a reliable enabler of circular construction within a decade.

The fundamental insight remains: the question is not whether we can detect deterioration, but whether we can characterize materials with the quantified reliability that structural design demands. Answering that question affirmatively is both technically feasible and essential for the transition to a circular built environment.

%=============================================================================
% References
%=============================================================================
\bibliographystyle{plainnat}
\bibliography{NDT_Review_references}

\end{document}
